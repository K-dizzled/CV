\documentclass[11pt,a4paper]{moderncv}

% moderncv themes
\moderncvtheme[grey]{classic}
\usepackage[utf8]{inputenc}                   % replace by the encoding you are using
\usepackage{xcolor}

% adjust the page margins
\usepackage[scale=0.88]{geometry}
%\setlength{\hintscolumnwidth}{3cm}						% if you want to change the width of the column with the dates
%\AtBeginDocument{\setlength{\maketitlenamewidth}{6cm}}  % only for the classic theme, if you want to change the width of your name placeholder (to leave more space for your address details
\AtBeginDocument{\recomputelengths}                     % required when changes are made to page layout lengths

% personal data
\firstname{Andrei}
\familyname{Kozyrev}
\title{Curriculum Vitae}
\email{kozyrev.andreiii2016@gmail.com}
\phone{+49 176 84270634}


%\nopagenumbers{}                             % uncomment to suppress automatic page numbering for CVs longer than one page


%----------------------------------------------------------------------------------
%            content
%----------------------------------------------------------------------------------
\begin{document}
\maketitle
\vspace{-10mm}
\section{Education}
\cventry{2018--2020}{SPbU D. K. Fadeev Academic Gymnasium}{Saint-Petersburg State University}{Saint-Petersburg, Russia}
{Applied Mathematics and informatics program}{}
\cventry{2020--2022}{The Faculty of Mathematics and Computer Science, Modern Software Engineering}{Saint-Petersburg State University}{Saint-Petersburg, Russia}
{}{}
\cventry{2022--\quad \quad}{Jacobs University Bremen, Computer Science}{Jacobs University Bremen}{Bremen, Germany}
{Currently a final year undergraduate student}{}
\subsection{Other}

\cvlistitem{Computer Science Center Introduction to IOS development course, Saint-Petersburg, Russia, Autumn, 2021.}
\cvlistitem{IOS Mobile Development 2-week intensive course in Educational Center in Russia}

\section{Academic Conferences}
\cventry{April~2020}{Scientific and Practical Conference ``Academic Gymnasium''}{Saint-Petersburg}{}{Russia}{Second place taken with ``Diser'' project}

\section{Work Experience}
\cventry{Summer 2022}{Junior Swift Developer}{Saint-Petersburg, Russia}{}{Russia}{Worked as a junior Swift developer for two months at a tech outsourcing company.}

\section{Selected Projects}
\cventry{}{A note}{\textit{names are referal to github projects}}{}{}{}
\cventry{2022}{\href{https://github.com/K-dizzled/lab1-k-dizzled-sokoban-rust.git}{Rust-NEAR-Sokoban}}{Sokoban game implemented in Rust using near blockchain.}{\textit{\textbf{Rust, near sdk}}}{}{}
\cventry{2022}{\href{https://github.com/K-dizzled/swift-high-performance-network-programming}{High-Performance Network Programming in Swift}}{Research on Highly-efficient network programming using Swift and any competitive ability of Swift in this sphere}{\textit{\textbf{Swift, Java, Python}}}{}{}
\cventry{2022}{\href{https://github.com/K-dizzled/progressive-gan-pytorch}{Progressive GAN Pytorch}}{Progressive generative adversarial network implementation from ``Progressive growing of GANs for improved Quality, Stability, and Variation'' paper}{\textit{\textbf{Python}}}{}{}
% \cventry{2022}{\href{https://github.com/K-dizzled/nQueens-puzzle-app-matlog-hw2}{N-Queens puzzle}}{An IOS app that solves how to place N queens on a N by N chessboard so that no two queens threaten each other. Puzzle is solved via generating a CNF formula, ecrypting the correct arrangement and running a SAT-solver.}{\textit{\textbf{Swift}}}{}{}
\cventry{2021}{\href{https://github.com/maxidorov/Football-App}{Football App}}{A fully working IOS app for monitoring soccer events. Has wide functionality: following players, following teams, adding events to calendar. Created in a team in Sirius as a PET-project}{\textit{\textbf{Swift}}}{}{}
% \cventry{2021}{\href{https://github.com/K-dizzled/HindleyMilnerAlgorithm}{Hindley Milner Algorithm}}{An implementation of the Hindley-Milner type inference algorithm}{\textit{\textbf{Haskell}}}{}{}
\cventry{2021}{\href{https://github.com/bot-mne-v-rot/p2beer}{P2Beer}}{A fully decentralized peer to peer desktop console messenger with a curses like interface. Created in a team of three people on first bachelor year}{\textit{\textbf{Kotlin}}}{}{}
% \cventry{2021}{\href{https://github.com/K-dizzled/ToDoApp}{ToDoApp}}{A mobile IOS app for managing tasks}{\textit{\textbf{Swift}}}{}{}
% \cventry{2021}{\href{https://github.com/K-dizzled/Huffman_coder}{Huffman coder}}{Huffman encoder and decoder with tests}{\textit{\textbf{C++}}}{}{}
% \cventry{2020}{\href{https://github.com/K-dizzled/C-Lang-HW/tree/master/hw_01}{BMP Editor}}{BMP format image editor for cropping and rotating images}{\textit{\textbf{C}}}{}{}
% \cventry{2020}{\href{https://github.com/K-dizzled/text-index-project}{Text Index}}{Console app for building a text-index using a trie data structure for online requests on text data}{\textit{\textbf{Kotlin}}}{}{}
% \cventry{2020}{\href{https://github.com/K-dizzled/hamming-code-linal-project}{Hamming code}}{Console app implementing hamming code}{\textit{\textbf{Kotlin}}}{}{}
% \cventry{2020}{\href{https://github.com/K-dizzled/DiserRepo}{Diser}}{A messenger, adapted for visually impared people. Created for the Conference, mentioned in Academic Conferences. I could have said about it in 
% a more detailed way, but the website I made already has all the information. The website link is below}{\textit{\textbf{Java}}}{}{}
% \cventry{2019}{\href{https://k-dizzled.github.io/delivery-food/}{Delivery Food}}{A website for a food delivery service. All the data about products is stored in Firebase database and loads directly from there}{\href{https://github.com/K-dizzled/delivery-food}{\textcolor{lightgray}{Link to github}}}{\textit{\textbf{HTML, CSS, JavaScript}}}{}

\section{Recent contests}
\cvlistitem{Place 16 globally in Reply Code Challenge 2021.}
\cvlistitem{Place 407 globally in Google Hash Code 2021.}

\section{Technical Strengths}
\cvlistitem{\textbf{Languages:} Swift, Rust, C++, Kotlin, Python, Haskell, SQL.}
\cvlistitem{\textbf{Skills:} Advanced Git skills, Experience with Coq, Knowledge of Algorithms and Data Structures}

% \section{Soft skills}
% \cvlistitem{``Maxed out his social skills''. Always able to find a common ground with anyone.}
% \cvlistitem{Team player. Able to team up a bunch of shy programmers together.}
% \cvlistitem{World class player in searching for data. Professional Googler. There is a saying ``Once on the Internet, always on the Internet''. It's
% similar for me. Once on the Internet, I can find it then.}

% \section{Languages}
% \cvline{Russian}{\small Native speaker.}
% \cvline{English}{\small C1 level.}
% \cvline{French}{\small Beginner level.}

\end{document}

%%% Local Variables:
%%% mode: latex
%%% TeX-master: t
%%% End:
