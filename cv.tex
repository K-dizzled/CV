\documentclass[11pt,a4paper]{moderncv}

% moderncv themes
\moderncvtheme[grey]{classic}
\usepackage[utf8]{inputenc}                   % replace by the encoding you are using
\usepackage{xcolor}

% adjust the page margins
\usepackage[scale=0.88]{geometry}
%\setlength{\hintscolumnwidth}{3cm}						% if you want to change the width of the column with the dates
%\AtBeginDocument{\setlength{\maketitlenamewidth}{6cm}}  % only for the classic theme, if you want to change the width of your name placeholder (to leave more space for your address details
\AtBeginDocument{\recomputelengths}                     % required when changes are made to page layout lengths

% personal data
\firstname{Andrei}
\familyname{Kozyrev}
\title{Curriculum Vitae}
\email{kozyrev.andreiii2016@gmail.com}


%\nopagenumbers{}                             % uncomment to suppress automatic page numbering for CVs longer than one page


%----------------------------------------------------------------------------------
%            content
%----------------------------------------------------------------------------------
\begin{document}
\maketitle
\vspace{-10mm}
\section{Education}
\cventry{2018--2020}{SPbU D. K. Fadeev Academic Gymnasium}{Saint-Petersburg State University}{Saint-Petersburg, Russia}
{Applied Mathematics and informatics program}{}
\cventry{2020--\quad\quad}{The Faculty of Mathematics and Computer Science, Modern Software Engineering}{Saint-Petersburg State University}{Saint-Petersburg, Russia}
{Currently a second-year student}{}
\subsection{Other}
%\cvlistitem{Short course in methods, tools and systems for information security,   Moscow Engineering Physics Institute, Moscow, Russia, December 2004.}
%\cvlistitem{Short course in hardware and software for computer networking, 
%Southern Federal University, Rostov-on-Don, Russia, November--December 2010.}
\cvlistitem{Computer Science Center Introduction to IOS development course, Saint-Petersburg, Russia, Autumn, 2021.}
\cvlistitem{IOS Mobile Development 2-week intensive course by Yandex in Educational Center ``Sirius''}
%\cvlistitem{Short course in iOS-applications Development, National Research University «MPEI», Moscow, Russia, June 14--19, 2015.}
%\cvlistitem{Online courses (coursera.org, edx.org) in Automata, Logic, Complilers, Machine Learning, Algorithms, Cryptography, Functional Programming, etc.)}


% \section{Master thesis}
% \cvline{title}{\emph{Singular-type operators in the spaces of vector-valued functions.}}
% \cvline{supervisor}{Prof. V.S. Pilidi}
% \cvline{description}{\small Algebras of operators of singular and bisingular type are considered. For these algebras, 
% the notions of an envelope of a family of operators are introduced, 
% necessary and sufficient conditions of its existence and uniqueness are given. 
% There are obtained estimates for the essential norm of the envelope operator. 
% The statements of the paper are demonstrated for the case of singular integral 
% operators with continuous operator-valued coefficients.}

\section{Academic Conferences}
\cventry{April~2020}{Scientific and Practical Conference ``Academic Gymnasium''}{Saint-Petersburg}{}{Russia}{}

\section{Software Development Experience}
\cventry{}{A note}{\textit{names are referal to github projects}}{}{}{}
\cventry{2021}{\href{https://github.com/maxidorov/Football-App}{Football App}}{A fully working IOS app for monitoring soccer events. Has wide functionality: following players, following teams, adding events to calendar. Created in a team in Sirius as a PET-project}{\textit{\textbf{Swift}}}{}{}
\cventry{2021}{\href{https://github.com/bot-mne-v-rot/p2beer}{P2Beer}}{A fully decentralized peer to peer desktop console messenger with a curses like interface. Created in a team of three people on first bachelor year}{\textit{\textbf{Kotlin}}}{}{}
\cventry{2021}{\href{https://github.com/K-dizzled/ProArtGan}{ProArtGan}}{Progressive generative adversarial network for generating landscape images}{\textit{\textbf{Python}}}{}{}
\cventry{2021}{\href{https://github.com/K-dizzled/ToDoApp}{ToDoApp}}{A mobile IOS app for managing tasks}{\textit{\textbf{Swift}}}{}{}
\cventry{2021}{\href{https://github.com/K-dizzled/Huffman_coder}{Huffman coder}}{Huffman encoder and decoder with tests}{\textit{\textbf{C++}}}{}{}
\cventry{2020}{\href{https://github.com/K-dizzled/C-Lang-HW/tree/master/hw_01}{BMP Editor}}{BMP format image editor for cropping and rotating images}{\textit{\textbf{C}}}{}{}
\cventry{2020}{\href{https://github.com/K-dizzled/text-index-project}{Text Index}}{Console app for building a text-index using a trie data structure for online requests on text data}{\textit{\textbf{Kotlin}}}{}{}
\cventry{2020}{\href{https://github.com/K-dizzled/hamming-code-linal-project}{Hamming code}}{Console app implementing hamming code}{\textit{\textbf{Kotlin}}}{}{}
\cventry{2020}{\href{https://github.com/K-dizzled/DiserRepo}{Diser}}{A messenger, adapted for visually impared people. Created for the Conference, mentioned in Academic Conferences. I could have said about it in 
a more detailed way, but the website I made already has all the information. The website link is below}{\textit{\textbf{Java}}}{}{}
\cventry{2020}{\href{https://k-dizzled.github.io/disermessaging/index.html}{Diser website}}{A website for a messenger mentioned above}{\href{https://github.com/K-dizzled/disermessaging}{\textcolor{lightgray}{Link to github}}}{\textit{\textbf{HTML, CSS, JavaScript}}}{}
\cventry{2019}{\href{https://k-dizzled.github.io/delivery-food/}{Delivery Food}}{A website for a food delivery service. All the data about products is stored in Firebase database and loads directly from there}{\href{https://github.com/K-dizzled/delivery-food}{\textcolor{lightgray}{Link to github}}}{\textit{\textbf{HTML, CSS, JavaScript}}}{}

\section{Technical Strengths}
\cvlistitem{Confident LaTeX knowledge. \href{https://github.com/bot-mne-v-rot}{\textcolor{lightgray}{Link to an organization with all study abstracts, made in TeX.}}}
\cvlistitem{Advanced Git skills}
\cvlistitem{Knowledge of Algorithms and Data Structures}

% \section{Selected Papers}
% \cvlistitem{V. Bragilevsky, V. S. Pilidi. Analog of Simonenko's theorem on an envelope of a family of operators for operators on the spaces of vector-valued functions.
% Izvestia vuzov. Severokavkazsky region. Vol. 4. 2005. (\emph{in Russian})}
% \cvlistitem{V. Bragilevsky. Limits of Folds Expressiveness. Practice of Functional Programming. Vol.~4. 2010. (\emph{in Russian})}

\section{Soft skills}
\cvlistitem{Always ready to learn.}
\cvlistitem{Always seeking challenging tasks and opportunities.}
\cvlistitem{``Maxed out his social skills''. Always able to find a common ground with anyone.}
\cvlistitem{Team player. Able to team up a bunch of shy programmers together.}
\cvlistitem{World class player in searching for data. Professional Googler. There is a saying ``Once on the Internet, always on the Internet''. It's
similar for me. Once on the Internet, I can find it then.}

\section{Languages}
\cvline{Russian}{\small Native speaker.}
\cvline{English}{\small C1 level.}
\cvline{French}{\small Beginner level.}

\end{document}

%%% Local Variables:
%%% mode: latex
%%% TeX-master: t
%%% End: